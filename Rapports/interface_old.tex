%%%%%%%%%%%%%%%%%%%%%%%%%%%%%%%%%%%%%%%%%
% University Assignment Title Page 
% LaTeX Template
% Version 1.0 (27/12/12)
%
% This template has been downloaded from:
% http://www.LaTeXTemplates.com
%
% Original author:
% WikiBooks (http://en.wikibooks.org/wiki/LaTeX/Title_Creation)
%
% License:
% CC BY-NC-SA 3.0 (http://creativecommons.org/licenses/by-nc-sa/3.0/)
% 
% Instructions for using this template:
% This title page is capable of being compiled as is. This is not useful for 
% including it in another document. To do this, you have two options: 
%
% 1) Copy/paste everything between \begin{document} and \end{document} 
% starting at \begin{titlepage} and paste this into another LaTeX file where you 
% want your title page.
% OR
% 2) Remove everything outside the \begin{titlepage} and \end{titlepage} and 
% move this file to the same directory as the LaTeX file you wish to add it to. 
% Then add \input{./title_page_1.tex} to your LaTeX file where you want your
% title page.
%
%%%%%%%%%%%%%%%%%%%%%%%%%%%%%%%%%%%%%%%%%
%\title{Title page with logo}
%----------------------------------------------------------------------------------------
%	PACKAGES AND OTHER DOCUMENT CONFIGURATIONS
%----------------------------------------------------------------------------------------

\documentclass[12pt]{article}

\usepackage[francais]{babel}
\usepackage[utf8x]{inputenc}
\usepackage[T1]{fontenc}
\usepackage{color}

\usepackage{amsmath}
\usepackage{graphicx}
\usepackage{enumerate}
\usepackage{url}
\usepackage{listings}

\usepackage{xcolor}
\colorlet{keyword}{blue!100!black!80}
\colorlet{comment}{green!90!black!90}
\lstdefinestyle{vhdl}{
   language     = VHDL,
   basicstyle   = \ttfamily,
   keywordstyle = \color{keyword}\bfseries,
   commentstyle = \color{comment}
}

\lstdefinestyle{equ}{
   basicstyle   = \ttfamily,
   keywordstyle = \color{keyword}\bfseries,
   commentstyle = \color{comment}
}

% Define new command
\newcommand{\HRule}{\rule{\linewidth}{0.5mm}}

\newcommand{\crt}{\emph{Nexys 4 DDR\ }}
%\def\thesubsection{\alph{section}}

\begin{document}

\begin{titlepage}

\center % Center everything on the page
 
%----------------------------------------------------------------------------------------
%	HEADING SECTIONS
%----------------------------------------------------------------------------------------

\textsc{\LARGE Universit\'e Pierre et Marie Curie}\\[1.5cm] % Name of your university/college
\textsc{\Large PSESI}\\[0.5cm] % Major heading such as course name

%----------------------------------------------------------------------------------------
%	TITLE SECTION
%----------------------------------------------------------------------------------------

\HRule \\[0.4cm]
{ \huge \bfseries Projet Centrale DCC}\\[0.4cm] % Title of your document
{ \huge \bfseries Document Définition Interface et Logique Enclenchement}\\[0.4cm] % Title of your document
\HRule \\[1.5cm]
 
%----------------------------------------------------------------------------------------
%	AUTHOR SECTION
%----------------------------------------------------------------------------------------

\begin{minipage}{0.4\textwidth}
\begin{flushleft} \large
\emph{\'Etudiant:}\\
Maxime \textsc{AYRAULT} 3203694 % Your name
\end{flushleft}
\end{minipage}
~
\begin{minipage}{0.4\textwidth}
\begin{flushright} \large
\emph{Encadrant:} \\
Julien \textsc{DENOULET} % Supervisor's Name
\end{flushright}
\end{minipage}\\[2cm]

%----------------------------------------------------------------------------------------
%	DATE SECTION
%----------------------------------------------------------------------------------------

{\large \today}\\[2cm] % Date, change the \today to a set date if you want to be precise

%----------------------------------------------------------------------------------------
%	LOGO SECTION
%----------------------------------------------------------------------------------------

%%\begin{figure}
%%  \subfigure[]{\includegraphics[scale=0.2\textwidth]{logo.png}} 
%%\end{figure} 
\includegraphics[width=0.2\textwidth]{logo.png}

%----------------------------------------------------------------------------------------

\vfill % Fill the rest of the page with whitespace

\end{titlepage}



%\begin{abstract}
%Your abstract here.
%\end{abstract}

\section{Introduction}
\label{sec:introduction}
Présentation des 3 interfaces~:
\begin{itemize}
\item Interface Homme Machine
\item Interface Centrale DCC et le circuit ferroviaire
\item Interface interne Centrale DCC - Moteur IXL
\end{itemize}

et présentation des règles de la logique d'enclenchement.


\section{Interface Homme Machine}
\label{sec:int-dcc}
Sur la carte
\begin{itemize}
\item Bouton gauche -> changement du paramètre
\item Bouton droit  -> changement de la valeur du paramètre
\end{itemize}

\hrulefill

Dans le design
\begin{enumerate}
\item en sortie :
  \begin{itemize}  
  \item les 7 selecteurs de segment Ca -> Cg
  \item les 8 selecteurs de digit   An (7 downto 0)
  \item octet adresse \textbf{\emph{(addr or aigui)}}  dans ADD
  \item octet speed \textbf{\emph{(speed)}}  dans SPEED
  \item octet fonction \textbf{\emph{(feat)}}  dans FCTN    
  \end{itemize}  
\item en entrée :
  \begin{itemize}  
  \item une horloge à 100 MHz
  \item un reset synchrone actif haut 
  \item la valeur du bouton gauche BTNL
  \item la valeur du bouton droit BTNR    
  \end{itemize}  
\end{enumerate}

\subsection{\underline{Interface commande train}}
\label{sec:ihm-train}

\bigskip

8 vitesses
\begin {enumerate}
\item vitesse 0 -> \emph{``01100000''}
\item vitesse 1 -> \emph{``01100010''}
\item vitesse 2 -> \emph{``01100100''}
\item vitesse 3 -> \emph{``01101000''}
\item vitesse 4 -> \emph{``01101010''}
\item vitesse 5 -> \emph{``01110000''}
\item vitesse 6 -> \emph{``01110100''}
\item vitesse 7 -> \emph{``01111111''}
\end {enumerate}  

\bigskip

6 adresses de train
\begin {enumerate}
\item adresse 0 -> \emph{``00000000''}  
\item adresse 1 -> \emph{``00000001''}
\item adresse 2 -> \emph{``00000010''}
\item adresse 3 -> \emph{``00000011''}
\item adresse 4 -> \emph{``00000100''}
\item adresse 5 -> \emph{``00000101''}
\item adresse 6 -> \emph{``00000110''}
\end {enumerate}  

\bigskip

8 adresses de d'aiguillage
\begin {enumerate}
\item elles ne sont pas encore définies. mais de 10 à 18 pour faire simple.
\end {enumerate}  

\bigskip

5 fonctions
\begin {enumerate}
\item phares on     -> \emph{``10010000''}
\item bruit moteur  -> \emph{``10000001''}
\item klaxon 1 long -> \emph{``10000100''}
\item klaxon 1 bref -> \emph{``10100100''}
\item bruit du vent -> \emph{``10100001''}
\end {enumerate}  

\bigskip

\subsection{\underline{Interface itinéraire}}
\label{sec:ihm_iti}

Elle n'a pas été implémentée, mais la base sur laquelle j'étais partie
\'etait d'affecter un switch de la carte par itinéraire.
Un itineraire correspondant à une suite de changement de positions
d'aiguillages et passage sur certains capteurs.

\newpage

\section{Interface Centrale DCC - Train, aiguillage et capteurs}
\label{sec:int_dcc}

\subsection{\underline{Interface Train}}
\label{sec:int_train}

\bigskip
Trames envoyées 4 par 4 séparées chacune par 400 us avec comme valeur 0.
\begin{center}
\begin{tabular}{|l|l|l|}
  \hline
  1er trame & IDLE \\
  \hline
  2eme trame & Vitesse \\
  \hline
  3eme trame & Action \\
  \hline
  4eme trame & IDLE \\
  \hline
\end{tabular}
\end{center}

\medskip
Les valeurs des trames à envoyer sont les mêmes jusqu'à un appui sur
le bouton central qui change la valeur courante des différentes
commandes.


\medskip
\begin{center}
\begin{tabular}{lll}
adresse & => & ``11111111'' \\
donnée  & => & ``00000000'' \\
CRC     & => & ``11111111'' \\
\end{tabular}
\end{center}

\medskip
Paramètres à envoyer stockés dans \emph{Master}.

Les différentes commandes viennent de l'IHM \emph{(voir
  \cite{rapport})}  et remplacent la valeur IDLE \emph{(si il y a eu
  un reset)}et sont envoyées en continue tant qu'il n'y a pas de nouvelles pression sur bouton Central.


\subsection{\underline{Interface Capteurs}}
\label{sec:int_cap}

A garder??

bla bla bla...


\newpage
\section{\underline{Interface Centrale DCC - Moteur IXL}}
\label{sec:int_ixl}   

La logique d'enclenchement est normalement basée sur les notions de signaux, de
circuit de voie et d'itinéraire ferroviaire. Dans le cadre du projet,
la logique d'enclenchement sera simplifiée et sera basée uniquement sur les
notions de circuits de voie et d'aiguillages. 

\medskip
Un \emph{Ciruit de Voie} (CdV) représente la plus petite partie de
voie sur lequel il est possible de détecter la présence d'un
train. Pour un train réel, la détection de la présence d'un train
s'effectue par les essieux du train. En effet, les essieux du train
sont conducteur (métal), ils peuvent servir comme un
\emph{interupteur} entre les 2 rails de la voie. Si une partie du
train est présent alors les 2 rails sont au même potentiel (le circuit
est dit fermé), sinon les rails sont a des potentiels différents (le
circuit est dit ouvert).

\medskip
Pour le projet, un Circuit de voie sera délimité par 2 capteurs de
passage(sauf en zone d'aiguille où le circuit de voie sera délimité
par les capteurs de chaque branche). L'état d'occupation du CdV sera
fourni pas le sens de parcours identifié par le capteur pour chaque
train.

\medskip
Un \emph{aiguillage} est un équipement de voie permettant de faire
circuler un train d'une voie à une autre. Une aiguille est constituée
d'une pointe et de deux branches. Selon la commande reçue par
l'aiguillage, la position de son aiguille change pour permettre au train
de circuler :
\begin{itemize}
\item sur la branche commandée en venant de la pointe de l'aiguillage
\item vers la pointe en venant de la branche commandée. 
\end{itemize}

\medskip
Dans tous les autre cas (aiguille entrebaillée ou en mauvaise position), 
il y a un risque de déraillement du train. Il existe également un autre 
risque autour des aiguillages, c'est le risque de collision entre deux
trains provenant de branches différentes en même temps. Le système 
d'enclenchement doit permettre d'éviter ces deux risques.
Dans un système d'enclenchement, une aiguille est représenté par une 
machine à états. Les états possibles sont:

\begin{itemize}
\item \emph{droite}. L'aiguille est positionnée du côté droit en regardant
l'aiguillage depuis la pointe
\item \emph{gauche}. L'aiguille est positionnée du côté gauche en regardant
l'aiguillage depuis la pointe
\item \emph{decontrôlée}. L'aiguille n'est positionnée ni du côté droit, 
ni du coté gauche (aiguille en mouvement ou Hors Service).
\end{itemize}

\medskip
L'interface entre la Centrale DCC et le Moteur IXL est bi-directionnelle. 

\begin{itemize}
\item Dans le sens Centrale vers IXL, les informations envoyées sont~:
  \begin{itemize}
  \item les demandes de commande d'aiguillage provenant de l'opérateur
  \item l'état des aiguilles (gauche/droite)
  \item l'information des capteurs de passage
  \end{itemize}

\item Dans le sens IXL vers Centrale, les informations envoyées sont~:
  \begin{itemize}
  \item les autorisations de commande des aiguilles
  \item l'état des circuits de voie pour faciliter la mise au point 
  du système d'enclenchement
  \end{itemize}
\end{itemize}


\subsection{\underline{Demande de commande des aiguillages}}
\label{sec:ixl_dem_aig}

L'opérateur peut demander de commander un aiguillage dans une certaine
position pour permettre au train de changer de voie.

Une demande de commande d'un aiguillage sera représenté par
1 bit avec la convention suivante~:
\begin{itemize}
\item 0 = demande de commande de aiguille à droite
\item 1 = demande de commande de aiguille à gauche
\end{itemize}
 
Il y a 8 aiguilles sur le circuit. L'indice du tableau correspond au
numéro de l'aiguille.

\medskip
La déclaration VHDL
\begin{lstlisting}[style=vhdl]
  
  signal Sw_Cmd_Req :   STD_LOGIC_VECTOR (7 downto 0);
  
\end{lstlisting}


\subsection{\underline{\'Etat des aiguillages}}
\label{sec:ixl_aig}

Il n'y a pas de capteur fournissant la position d'une aiguille sur la
maquette.  Par conséquent, l'état d'un aiguillage correspondra à l'état
de la dernière commande envoyée à l'aiguillage depuis la Centrale DCC.
L'état \emph{décontrôlé} ne sera pas utilisé.
 
L'état d'un aiguillage sera représenté par 1 bit avec la convention
suivante~:
\begin{itemize}
\item 0 = aiguille positionnée à droite
\item 1 = aiguille positionnée à gauche
\end{itemize}
 
Il y a 8 aiguilles sur le circuit. L'indice du tableau correspond au
numéro de l'aiguille.

\medskip
La déclaration VHDL
\begin{lstlisting}[style=vhdl]

  signal Sw_State   :   STD_LOGIC_VECTOR (7 downto 0);
  
\end{lstlisting}


\subsection{\underline{\'Information capteur de passage}}
\label{sec:ixl_cdv}

Des capteurs de passage sont positionnés au long de la voie pour 
permettre de délimiter les Circuits de Voie. Un capteur de passage
fournit l'adresse du train venant de passer sur le capteur ainsi que
sa direction. La direction est une valeur arbitraire associée à 
chaque capteur pour connaitre le sens de parcours du train sur le capteur.


Il y a 6 trains au maximum. Les capteurs sont codés comme suit~:
\begin{itemize}
  \item dir = 0b01 : train dans le sens Up
  \item dir = 0b10 : train dans le sens Down
  \item dir = 0b11 : aucun train
  \item addr = 0x01 à 0x06 : numéro du train 1 à 6 passant sur le capteur
\end{itemize}  

Il y a 20 capteurs de passage sur la maquette de train. L'indice du
tableau correspond au numéro du capteur de passage.

\medskip
La déclaration VHDL
\begin{lstlisting}[style=vhdl]

  type sensor is
    record
      addr : STD_LOGIC_VECTOR (7 downto 0) ;
      dir  : STD_LOGIC_VECTOR (1 downto 0) ;
    end record ;

  -- Sensors state type
  type SE_state is array (31 downto 0) of sensor;

\end{lstlisting}


\subsection{\underline{Autorisation de commande des aiguillages}}
\label{sec:ixl_dem_aig}

Une fois que le système d'enclenchement a vérifé que toutes les
conditions requises pour bouger une aiguille sont remplies, il 
retourne à la Centrale DCC son autorisation de commande des aiguillages.

L'autorisation de commande d'un aiguillage sera représenté par
1 bit avec la convention suivante~:
\begin{itemize}
\item 0 = autorisation de commande de aiguille à droite
\item 1 = autorisation de commande de aiguille à gauche
\end{itemize}
 
Il y a 8 aiguilles sur le circuit. L'indice du tableau correspond au
numéro de l'aiguille.

\medskip
La déclaration VHDL
\begin{lstlisting}[style=vhdl]

  Sw_Cmd_Aut : out STD_LOGIC_VECTOR (7 downto 0);

\end{lstlisting}


  
\subsection{\underline{\'Etat des Circuits de Voie}}
\label{sec:st_sig}

Chaque circuit de voie possède 2 états; \emph{Libre} ou
\emph{Occupé}. De façon à pouvoir effectuer la gestion des trains, il
est nécessaire d'identifier quel train occupe le circuit de voie.  Il
y a 6 trains au maximum. Les états sont codés come suit~:
\begin{itemize}
  \item libre : 0x07
  \item occupé : numéro du train 0x01 à 0x06
\end{itemize}  

Il y a 16 circuits de voie sur la maquette de train. L'indice du
tableau correspond au numéro du Circuit de Voie.

\medskip
La déclaration VHDL
\begin{lstlisting}[style=vhdl]

  --debug output
  TC_out     : out STD_LOGIC_VECTOR (7 downto 0)

\end{lstlisting}




\newpage
\section{\underline{Logique Enclenchement}}
\label{sec:log_enc}

Cette section présente la logique d'enclenchement, sa justification et
les conversions de nommage utilisées dans les équations.

\subsection{\underline{Circuit de Voie}}
\label{sec:CdV}

Un Circuit de Voie est la section de voie se trouvant entre 2 capteurs
de présence consécutifs. Un capteur de présence permet de
connaître, le numéro du train et son sens de déplacement. Le sens de
déplacement va permettre de savoir quel Circuit de voie est occupé par le
train. Deux directions sont arbitrairement déclarées pour chaque
capteur; Up et Down.

Convention de nommage des éléments :
\begin{itemize}
\item $SE\_Up_{nn}$, 1 lorsqu'un train vient de passer le capteur numéro
  nn dans le sens Up
\item $SE\_Do_{nn}$, 1 lorsqu'un train vient de passer le capteur numéro
  nn dans le sens Down
\item $TC_{nn}$, 1 lorsque le Circuit de Voie est libre (aucun train
  présent sur le CdV), 0 lorsque le circuit de voie est occupée. Cette
  logique va dans le sens de la sécurité, en cas de perte de courant
  ou de dysfonctionnement, il n'est pas possible de savoir si le
  Circuit De Voie est occupé ou libre, donc on choisit la valeur la
  plus sécuritaire, occupé même si aucun train n'est présent.
\end{itemize}

\'A l'initialisation, tous les Circuits de Voie sont libres.

\medskip
Le Circuit de voie $TC_{nn}$ est occupé si une des condition suivantes
est vérifiée~:
\begin{itemize}
\item le capteur $SE\_Up_{nn}$ passe à 1, un train vient de rentrer dans
  le sens Up
\item le capteur $SE\_Do_{mm}$ passe à 1, un train vient de rentrer dans
  le sens Down
\item le Circuit de Voie est occupé et le train n'est pas sorti dans
  le sens Up ou Down
\end{itemize}

Il faut inverser cette logique car lorsque le Circuit de voie est
occupé, on cherche à avoir une valeur 0.

\medskip
L'équation générale d'occupation d'un Circuit de Voie est~:
$$\boxed{
  TC_{nn} \Leftarrow \neg SE\_Up_{nn} \land \neg SE\_Do_{mm} \land
  (TC_{nn} \lor SE\_Do_{nn} \lor SE\_Up_{mm});
}$$


\subsection{\underline{Signaux}}
\label{sec:esp}

Il existe deux types de signaux; les signaux d'espacement qui
garantissent que 2 trains ne vont pas se percuter et les signaux de
manoeuvre qui garantissent qu'un train ne va pas parcourir une
aiguille qui n'est pas contrôlée.

Convention de nommage des éléments :
\begin{itemize}
\item $SI_{nn}$, 1 lorsque le signal est Vert, 0 pour afficher un signal
  Rouge
\end{itemize}

\'A l'initialisation, tous les signaux sont au Vert.

Un Signal d'espacement passe~:
\begin{itemize}
\item au Rouge, lorsque le signal d'espacement ou le signal de
  manoeuvre est au Rouge
\item au Vert, lorsque le signal d'espacement et le signal de
  manoeuvre sont au Vert
\end{itemize}


\medskip
L'équation générale d'un signal est~:
$$\boxed{
  SI_{nn} \Leftarrow SI\_SP_{nn} \land SI\_IX_{nn};
}$$


\subsection{\underline{Signal d'espacement des trains}}
\label{sec:esp}

Il est dangereux d'avoir 2 trains dans le même Circuit de Voie car
rien ne permet de garantir que les 2 trains ne vont pas se
percuter. Il n'est pas certain qu'un conducteur puisse arrêter le
train devant un signal car la distance de freinage entre le signal et
le Circuit de Voie à protéger n'est pas toujours suffisante. C'est
pourquoi, il est nécessaire de toujours avoir un Circuit de Voie libre
entre 2 trains. Ce Circuit de Voie est nommé, \emph{CdV tampon}.  Dans
le cas de 2 trains qui se suivent, il est nécessaire que la longueur
du Circuit de Voie soit plus grande que la distance d'arrêt du train
suiveur.  Dans le cas de 2 trains en sens inverse, il est nécessaire
de s'assurer que la longueur du Circuit de Voie tampon est égale à 2
fois la longueur d'arrêt d'un train.

L'entrée dans un Circuit de Voie est protégée par un signal dit
\emph{d'espacement}.

Convention de nommage des éléments~:
\begin{itemize}
\item $SI\_SP_{nn}$, 1 lorsque le signal est Vert, 0 pour afficher un
  signal Rouge
\end{itemize}

\'A l'initialisation, tous les signaux sont au Vert.

Un Signal d'espacement passe~:
\begin{itemize}
\item au Rouge, lorsque le $TC_{nn}$ est occupé ou le $TC_{mm}$ est occupé
\item au Vert, lorsque les $TC_{nn}$ et $TC_{mm}$ sont libres.
\end{itemize}

\medskip
L'équation générale d'un signal d'espacement est~:
$$\boxed{
  SI\_SP_{nn} \Leftarrow TC_{nn} \land TC_{mm};
}$$


\subsection{\underline{Signal de manoeuvre}}
\label{sec:esp}

Un signal de manoeuvre garantit qu'un train peut traverser une zone
d'aiguillage sans danger. C'est à dire que les aiguilles de la zone
sont toutes enclenchées.




\subsection{\underline{Zone d'aiguille}}
\label{sec:aig}

Une aiguille peut être autorisée à bouger à droite (resp. gauche) 
si elle a reçu une commande à droite (resp. à gauche) et que le 
Circuit de Voie de l'aiguille et ses Ciruits de voie adjacents
sont libres. 

\medskip
L'équation générale d'autorisation de commande d'une aiguille est~:
$$\boxed{
  SW\_AUT\_RI_{nn} \Leftarrow SW\_CMD\_RI_{nn} \land TC_{nn} \land
  TC_{mm} ;
}$$
$$\boxed{
  SW\_AUT\_LE_{nn} \Leftarrow SW\_CMD\_LE_{nn} \land TC_{nn} \land TC_{mm} ;
}$$


\subsection{\underline{Itinéraire}}
\label{sec:iti}

Cette partie n'a pas été implémentée mais voici comment elle aurait dû
fonctionner.

Chaque itinéraire est un tableau de n instructions.
Le tableau contient l’état des aiguilles et la valeur des Tracks
circuits  occupés, quand les deux  valeurs sont identiques, il faut
passer  à la case suivante du  tableau.

Il y a 3 états pour chauques aiguillages,
\begin{itemize}
  \item '0' Pour forcer un aiguillage à être fermé.
  \item '1' Pour forcer un aiguillage à être ouvert.
  \item 'DC' La valeur ne nous interesse pas.    
\end{itemize}  

Idem pour les Tracks Circuits

Le but est de tenter d’obtenir les deux valeurs en même temps pour
passer à la case suivante du tableau, une fois l’itinéraire complété,
on  allume la led correspondante au switch pour signaler la fin de
l’itinéraire. 



\section{Logique d'enclenchement}
\label{sec:logic}
\subsection{\underline{Occupation de CdV}}
\label{sec:cdv_occ}

Voici le tableau qui permet de connaitre l'état d'un Track Circuit
selon l'état des capteurs.

\begin{center}
\begin{tabular}{|l||llll|}
  \hline
  TC\_01 & SE\_UP\_01 & SE\_UP\_19 & SE\_DO\_02 & SE\_DO\_08 \\
  \hline
  TC\_02 & SE\_UP\_02 & SE\_DO\_03 & & \\
  \hline
  TC\_03 & SE\_UP\_03 & SE\_DO\_04 & & \\
  \hline
  TC\_04 & SE\_UP\_04 & SE\_UP\_10 & SE\_DO\_05 & SE\_DO\_20 \\
  \hline
  TC\_05 & SE\_UP\_05 & SE\_DO\_06 & & \\
  \hline
  TC\_06 & SE\_UP\_06 & SE\_DO\_01 & & \\
  \hline
  TC\_07 & SE\_UP\_08 & SE\_DO\_09 & & \\
  \hline
  TC\_08 & SE\_UP\_09 & SE\_DO\_10 & & \\
  \hline
  TC\_09 & SE\_UP\_11 & SE\_DO\_12 & & \\
  \hline
  TC\_10 & SE\_UP\_12 & SE\_DO\_07 & & \\
  \hline
  TC\_11 & SE\_UP\_07 & SE\_UP\_13 & SE\_DO\_14 & SE\_DO\_19 \\
  \hline
  TC\_12 & SE\_UP\_14 & SE\_DO\_15 & & \\
  \hline
  TC\_13 & SE\_UP\_15 & SE\_DO\_16 & & \\
  \hline
  TC\_14 & SE\_UP\_16 & SE\_UP\_20 & SE\_DO\_11 & SE\_DO\_17 \\
  \hline
  TC\_15 & SE\_UP\_17 & SE\_DO\_18 & & \\
  \hline
  TC\_16 & SE\_UP\_18 & SE\_DO\_13 & & \\
  \hline
\end{tabular}
\end{center}

Un Track Circuit est considéré comme occupé lorsque l'un des capteurs
renvoi une information qui permet de faire changer l'état du Track
Circuit dans le tableau ci dessus.


\subsection{\underline{Mouvement d'aiguille}}
\label{sec:mvt_aig}

Voici le tableau qui regoupe les différentes équations qui vont
permettre de definir l'état de chacune des 8 aiguilles.


\begin{center}
\begin{tabular}{|l||lllll|}
  \hline
  SW1 & TC\_01 & TC\_02 & TC\_06 & TC\_07 & TC\_11 \\
  \hline
  SW2 & TC\_01 & TC\_02 & TC\_06 & TC\_07 & TC\_11 \\
  \hline
  SW3 & TC\_11 & TC\_01 & TC\_16 & TC\_10 & TC\_12 \\
  \hline
  SW4 & TC\_11 & TC\_01 & TC\_16 & TC\_10 & TC\_12 \\
  \hline
  SW5 & TC\_04 & TC\_03 & TC\_05 & TC\_08 & TC\_09 \\
  \hline
  SW6 & TC\_04 & TC\_03 & TC\_05 & TC\_08 & TC\_09 \\
  \hline
  SW7 & TC\_14 & TC\_04 & TC\_09 & TC\_15 & TC\_13 \\
  \hline
  SW8 & TC\_14 & TC\_04 & TC\_09 & TC\_15 & TC\_13 \\
  \hline
\end{tabular}
\end{center}

On obtient ainsi en sortie de notre module ce qu'il faut envoyer sur
chaque aiguillages pour eviter d'avoir deux trains se rentrant dedans.

\newpage

\bibliographystyle{plain}
\bibliography{biblio_inter}

\end{document}
