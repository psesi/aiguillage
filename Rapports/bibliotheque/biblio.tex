%%%%%%%%%%%%%%%%%%%%%%%%%%%%%%%%%%%%%%%%%
% University Assignment Title Page 
% LaTeX Template
% Version 1.0 (27/12/12)
%
% This template has been downloaded from:
% http://www.LaTeXTemplates.com
%
% Original author:
% WikiBooks (http://en.wikibooks.org/wiki/LaTeX/Title_Creation)
%
% License:
% CC BY-NC-SA 3.0 (http://creativecommons.org/licenses/by-nc-sa/3.0/)
% 
% Instructions for using this template:
% This title page is capable of being compiled as is. This is not useful for 
% including it in another document. To do this, you have two options: 
%
% 1) Copy/paste everything between \begin{document} and \end{document} 
% starting at \begin{titlepage} and paste this into another LaTeX file where you 
% want your title page.
% OR
% 2) Remove everything outside the \begin{titlepage} and \end{titlepage} and 
% move this file to the same directory as the LaTeX file you wish to add it to. 
% Then add \input{./title_page_1.tex} to your LaTeX file where you want your
% title page.
%
%%%%%%%%%%%%%%%%%%%%%%%%%%%%%%%%%%%%%%%%%
%\title{Title page with logo}
%----------------------------------------------------------------------------------------
%	PACKAGES AND OTHER DOCUMENT CONFIGURATIONS
%----------------------------------------------------------------------------------------

\documentclass[12pt]{article}

\usepackage[francais]{babel}
\usepackage[utf8x]{inputenc}
\usepackage[T1]{fontenc}
\usepackage{color}

\usepackage{amsmath}
\usepackage{graphicx}
\usepackage{enumerate}
\usepackage{url}
\usepackage{listings}

\usepackage{xcolor}
\colorlet{keyword}{blue!100!black!80}
\colorlet{comment}{green!90!black!90}
\lstdefinestyle{vhdl}{
   language     = VHDL,
   basicstyle   = \ttfamily,
   keywordstyle = \color{keyword}\bfseries,
   commentstyle = \color{comment}
}

\lstdefinestyle{equ}{
   basicstyle   = \ttfamily,
   keywordstyle = \color{keyword}\bfseries,
   commentstyle = \color{comment}
}

% Define new command
\newcommand{\HRule}{\rule{\linewidth}{0.5mm}}

\newcommand{\crt}{\emph{Nexys 4 DDR\ }}
%\def\thesubsection{\alph{section}}

\begin{document}

\begin{titlepage}

\center % Center everything on the page
 
%----------------------------------------------------------------------------------------
%	HEADING SECTIONS
%----------------------------------------------------------------------------------------

\textsc{\LARGE Universit\'e Pierre et Marie Curie}\\[1.5cm] % Name of your university/college
\textsc{\Large PSESI}\\[0.5cm] % Major heading such as course name

%----------------------------------------------------------------------------------------
%	TITLE SECTION
%----------------------------------------------------------------------------------------

\HRule \\[0.4cm]
{ \huge \bfseries Projet Centrale DCC}\\[0.4cm] % Title of your document
{ \huge \bfseries Document De recherche bibliothécaire}\\[0.4cm] % Title of your document
\HRule \\[1.5cm]
 
%----------------------------------------------------------------------------------------
%	AUTHOR SECTION
%----------------------------------------------------------------------------------------

\begin{minipage}{0.4\textwidth}
\begin{flushleft} \large
\emph{\'Etudiant:}\\
Maxime \textsc{AYRAULT} 3203694 % Your name
\end{flushleft}
\end{minipage}
~
\begin{minipage}{0.4\textwidth}
\begin{flushright} \large
\emph{Encadrant:} \\
Julien \textsc{DENOULET} % Supervisor's Name
\end{flushright}
\end{minipage}\\[2cm]

%----------------------------------------------------------------------------------------
%	DATE SECTION
%----------------------------------------------------------------------------------------

{\large \today}\\[2cm] % Date, change the \today to a set date if you want to be precise

%----------------------------------------------------------------------------------------
%	LOGO SECTION
%----------------------------------------------------------------------------------------

%%\begin{figure}
%%  \subfigure[]{\includegraphics[scale=0.2\textwidth]{logo.png}} 
%%\end{figure} 
\includegraphics[width=0.2\textwidth]{logo.png}

%----------------------------------------------------------------------------------------

\vfill % Fill the rest of the page with whitespace

\end{titlepage}



%\begin{abstract}
%Your abstract here.
%\end{abstract}

\section{Introduction}
\label{sec:introduction}

Le but de mon sujet était de développer sur FPGA un système
d'aiguillage automatisé pour centrale DCC sur un train miniature.

J'ai décidé de découper mon projet en deux étapes afin de bien dissocier les problèmes durant le projet.

\medskip

\begin{itemize}
  \item Création de la centrale DCC.
  \item Ajout de la gestion des aiguillages.
\end{itemize}  

\medskip

J'ai rencontré quelques problèmes bloquants, notamment au niveau de la
maquette  qui était incomplète et qui de ce fait ne remplissait pas
les conditions nécessaires  à la réalisation de mon projet. J'ai donc
décidé afin de pouvoir terminer ce  projet de passer en mode
simulation. Ce mode peut être considéré comme un mode test avant
l'application en mode  réel.

\section{Mots clefs retenus}
\label{sec:Mots_clefs}

j'ai retenu deux mots clés qui me semblent importants pour la création de la centrale DCC.
\textbf{FPGA} et \textbf{Protocole DCC}

\medskip

La \emph{carte FPGA} sert de plateforme de développement.

Le \emph{protocole DCC} est le protocole qui permet de communiquer des
informations aux trains.

\bigskip

Les mots clefs les plus importants concernant l'aiguillage sont
\textbf{interlocking} et \textbf{Track Circuit}.

\medskip

\emph{L'interlocking} est la façon d'empêcher plusieurs trains de se retrouver l'un en face de l'autre.

Un \emph{Track Circuit} est la plus petite portion de voie entre deux \emph{capteurs}.

\section{Descriptif Recherche}
\label{sec:descriptif}

Pour obtenir les différents documents dont j'ai eu besoin lors de mon
projet je suis passé par plusieurs sources. La principale étant
\emph{M.Denoulet} qui m'a donné les premiers documents \cite{FPGA1} \cite{DCC} \cite{Jouef}   dont j'avais besoin
pour comprendre le fonctionnement d'une \emph{centrale DCC}.

\smallskip

J'ai utilisé \emph{Google} en tant que moteur de recherche afin de
trouver divers documents \cite{Xilink} \cite{IXL} \cite{OCAML}
\cite{VHDL} \cite{siteferro} qui m'ont aider lors de mon projet.

\smallskip

Les documents complémentaires que j'ai dû écrire
pour notamment définir les règles à mettre en place m'ont permis de
me poser les bonnes questions lors de se projet \cite{rapport}.


\section{analyse de 3 sources}
\label{sec:analyse}

\emph{Source 1} :

\smallskip

Pour développer mon application Ocaml, je me suis appuyé sur le
document de référence du \textbf{Ocaml} \cite{OCAML}. Ce document se trouve sur
le site de l'\textbf{INRIA}. J'ai choisi ce site car le niveau de
fiabilité concernant Ocaml est élevé. j'ai trouvé dans ce document tous les éléments nécessaires au développement de mon application.

\bigskip

\emph{Source 2} :

\smallskip

\textbf{Datasheet Xilink} \cite{Xilink}. J'ai trouvé la datasheet de la
carte FPGA , la $NEXYS
4DDR$, que j'ai utilisé tout au long de mon projet, sur le site
officiel du fabricant  de la carte \textbf{XILINK}. Ce
document m'a permis de mieux comprendre le fonctionnement de la carte.
J'ai essayé d'exploiter le mieux possible les informations fournies.

\bigskip

\emph{Source 3} :

\smallskip

Pour déterminer la meilleure façon d'implémenter
\textbf{l'interlocking} dans le cadre de mon projet, je me suis documenté en allant sur
le site $www.railwaysignalling.eu$ \cite{IXL} qui regroupe toutes les
informations partagées sur le fonctionnement des trains en
Europe. Cela m'a permis d'obtenir des informations historiques mais
aussi des informations et explications sur certaines  notions
nécessaires à la réalisation  de mon projet.


\newpage

\bibliographystyle{plain}
\bibliography{biblio}

\end{document}
