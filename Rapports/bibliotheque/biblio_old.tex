%%%%%%%%%%%%%%%%%%%%%%%%%%%%%%%%%%%%%%%%%
% University Assignment Title Page 
% LaTeX Template
% Version 1.0 (27/12/12)
%
% This template has been downloaded from:
% http://www.LaTeXTemplates.com
%
% Original author:
% WikiBooks (http://en.wikibooks.org/wiki/LaTeX/Title_Creation)
%
% License:
% CC BY-NC-SA 3.0 (http://creativecommons.org/licenses/by-nc-sa/3.0/)
% 
% Instructions for using this template:
% This title page is capable of being compiled as is. This is not useful for 
% including it in another document. To do this, you have two options: 
%
% 1) Copy/paste everything between \begin{document} and \end{document} 
% starting at \begin{titlepage} and paste this into another LaTeX file where you 
% want your title page.
% OR
% 2) Remove everything outside the \begin{titlepage} and \end{titlepage} and 
% move this file to the same directory as the LaTeX file you wish to add it to. 
% Then add \input{./title_page_1.tex} to your LaTeX file where you want your
% title page.
%
%%%%%%%%%%%%%%%%%%%%%%%%%%%%%%%%%%%%%%%%%
%\title{Title page with logo}
%----------------------------------------------------------------------------------------
%	PACKAGES AND OTHER DOCUMENT CONFIGURATIONS
%----------------------------------------------------------------------------------------

\documentclass[12pt]{article}

\usepackage[francais]{babel}
\usepackage[utf8x]{inputenc}
\usepackage[T1]{fontenc}
\usepackage{color}

\usepackage{amsmath}
\usepackage{graphicx}
\usepackage{enumerate}
\usepackage{url}
\usepackage{listings}

\usepackage{xcolor}
\colorlet{keyword}{blue!100!black!80}
\colorlet{comment}{green!90!black!90}
\lstdefinestyle{vhdl}{
   language     = VHDL,
   basicstyle   = \ttfamily,
   keywordstyle = \color{keyword}\bfseries,
   commentstyle = \color{comment}
}

\lstdefinestyle{equ}{
   basicstyle   = \ttfamily,
   keywordstyle = \color{keyword}\bfseries,
   commentstyle = \color{comment}
}

% Define new command
\newcommand{\HRule}{\rule{\linewidth}{0.5mm}}

\newcommand{\crt}{\emph{Nexys 4 DDR\ }}
%\def\thesubsection{\alph{section}}

\begin{document}

\begin{titlepage}

\center % Center everything on the page
 
%----------------------------------------------------------------------------------------
%	HEADING SECTIONS
%----------------------------------------------------------------------------------------

\textsc{\LARGE Universit\'e Pierre et Marie Curie}\\[1.5cm] % Name of your university/college
\textsc{\Large PSESI}\\[0.5cm] % Major heading such as course name

%----------------------------------------------------------------------------------------
%	TITLE SECTION
%----------------------------------------------------------------------------------------

\HRule \\[0.4cm]
{ \huge \bfseries Projet Centrale DCC}\\[0.4cm] % Title of your document
{ \huge \bfseries Document De recherche bibliothécaire}\\[0.4cm] % Title of your document
\HRule \\[1.5cm]
 
%----------------------------------------------------------------------------------------
%	AUTHOR SECTION
%----------------------------------------------------------------------------------------

\begin{minipage}{0.4\textwidth}
\begin{flushleft} \large
\emph{\'Etudiant:}\\
Maxime \textsc{AYRAULT} 3203694 % Your name
\end{flushleft}
\end{minipage}
~
\begin{minipage}{0.4\textwidth}
\begin{flushright} \large
\emph{Encadrant:} \\
Julien \textsc{DENOULET} % Supervisor's Name
\end{flushright}
\end{minipage}\\[2cm]

%----------------------------------------------------------------------------------------
%	DATE SECTION
%----------------------------------------------------------------------------------------

{\large \today}\\[2cm] % Date, change the \today to a set date if you want to be precise

%----------------------------------------------------------------------------------------
%	LOGO SECTION
%----------------------------------------------------------------------------------------

%%\begin{figure}
%%  \subfigure[]{\includegraphics[scale=0.2\textwidth]{logo.png}} 
%%\end{figure} 
\includegraphics[width=0.2\textwidth]{logo.png}

%----------------------------------------------------------------------------------------

\vfill % Fill the rest of the page with whitespace

\end{titlepage}



%\begin{abstract}
%Your abstract here.
%\end{abstract}

\section{Introduction}
\label{sec:introduction}

Le but de mon sujer était de développer sur FPGA un systeme
d'aiguillage automatisé pour centrale DCC sur un train miniature.

J'ai descidé de decouper mon projet en deux étapes :
\begin{itemize}
  \item La création de la centrale DCC elle même.
  \item Ajouter la gestion des aiguillages à mon projet.
\end{itemize}  

J'ai été bloqué un moment et j'ai du réaliser mon sujet uniquement en
simulation car la maquette n'était pas complète.

\section{Mots clefs retenus}
\label{sec:Mots_clefs}

\textbf{FPGA} et \textbf{Protocole DCC} sont les deux mots clefs le plus important pour
la création de la centrale DCC.

\medskip

La \emph{carte FPGA} sert de plateforme de developpement, c'est sur elle que
va se dérouler mon projet.

Le \emph{protocole DCC} est le protocole qui permet de communiquer des
informations aux trains.

\bigskip

Du coté des aiguillages les mots clefs les plus important sont
\textbf{interlocking} et \textbf{Track Circuit}.

\medskip

\emph{L'interlocking} est la façon d'empêcher plusieurs trains de se rentrer
dedans.

Et un \emph{Track Circuit} est la plus petite portion de voie entre deux \emph{capteurs}.

\section{Descriptif Recherche}
\label{sec:descriptif}

Pour obtenir les différents documents dont j'ai eu besoin lors de mon
projet je suis passé par plusieurs sources. La principale étant
\emph{M.Denoulet} qui m'a donné les premiers documents \cite{FPGA1} \cite{DCC} \cite{Jouef}   dont j'avais besoin
pour comprendre le fonctionnement d'une \emph{centrale DCC}.

\smallskip

Je me suis aussi servit de \emph{Google} comme moteur de recherche afin de
trouver divers documents \cite{Xilink} \cite{IXL} \cite{OCAML}
\cite{VHDL} \cite{siteferro} qui m'ont aider lors de mon projet

\smallskip

Et je me suis aussi servit de certains documents que j'ai du écrire
lors de ce projet \cite{rapport}. 


\newpage

\section{analyse de 3 sources}
\label{sec:analyse}

\emph{Source 1} :

\smallskip

Pour développer mon application Ocaml, je me suis appuyé sur le
document de référence du \textbf{Ocaml} \cite{OCAML}. Ce document se touve sur
le site de l'\textbf{INRIA}, qui du coup donne à se document un niveau de
fiabilité plutôt élevé. Il m'a été utile pour développer mon
application, car dans se document se regroupe toutes les informations
consernant se langage.

\bigskip

\emph{Source 2} :

\smallskip

La \textbf{Datasheet Xilink} \cite{Xilink}. J'ai trouvé la datasheet de la
carte FPGA que j'ai utilisé tout au long de mon projet, la $NEXYS
4DDR$, sur le site officiel du fabricant de la carte \textbf{XILINK}. Ce
document m'a permis de mieux comprendre le fonctionnement de ma carte
et de pouvoir l'exploiter du mieux que j'ai pu.

\bigskip

\emph{Source 3} :

\smallskip

Pour comprendre les diférentes façon dont je pouvais implementer
\textbf{l'interlocking} dans le cadre de mon projet, je me suis documenté sur
le site $www.railwaysignalling.eu$ \cite{IXL} qui regroupe toutes les
informations partagées sur le fonctionnement des trains dans
l'europe. Pour le côté historique et explicatif de certaines notions
ce site à été très utile pour moi.


\newpage

\bibliographystyle{plain}
\bibliography{biblio}

\end{document}
